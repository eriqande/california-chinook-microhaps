%!TEX root = supplement.tex

%%%%%%%%%%%%%%%%%%%
\begin{figure}
\includegraphics[width=\textwidth]{images/lfar-assoc-faceted.jpg}
\caption[
	Negative log base 10 of association $p$-values for
	individual SNPs for late-fall versus fall run
]{
	\footnotesize Negative log base 10 of association $p$-values for
	individual SNPs for late-fall versus fall run.  $x$ axis shows position in genome (in megabases),
	with color alternating by chromosome, as indicated by numbers above the $x$-axis. ``Unk'' refers
	to unplaced scaffolds in the Otsh\_v1.0 genome assembly \citep{christensen2018chinook}. The 
	upper panel is the comparison between late-fall and Feather River Hatchery fall, while the lower 
	panel is the comparison of late-fall to San Joaquin River fall. 
}
\label{fig:lfar-assoc}
\end{figure}



%%%%%%%%%%%%%%%%%%%%%% 
\begin{figure}
\begin{center}
\includegraphics[width=0.7\textwidth]{images/num-alle-barplot.pdf}
\end{center}
\caption[Number of loci with different
total numbers of alleles]{\footnotesize Number of loci with different
total numbers of alleles in the data set.}
\label{fig:num-alle}
\end{figure}
%%%%%%%%%%%%%%%%%%%%




%%%%%%%%%%%%%%%%%%%%%% 
\begin{figure}
\begin{center}
\includegraphics[width=0.7\textwidth]{images/minor-modes-crop.pdf}
\end{center}
\caption[STRUCTURE minor modes found by CLUMPAK]{\footnotesize At each value of $K$
for which a minor mode was found, these plots show them}
\label{fig:minor-modes}
\end{figure}
%%%%%%%%%%%%%%%%%%%%


%%%%%%%%%%%%%%%%%%%%%% 
\begin{figure}
\begin{center}
\includegraphics[width=0.8\textwidth]{images/ass-table-80-crop.pdf}
\end{center}
\caption[Assignment table for fish with scaled likelihood $ > 0.8$]{\footnotesize Assignment table
like that in Fig.~\ref{fig:gsi}b in the paper, but constrained so that only fish assigning
to a reporting unit with scaled likelihood greater than 0.8 are included.}
\label{fig:eighty}
\end{figure}
%%%%%%%%%%%%%%%%%%%%





%%%%%%%%%%%%%%%%%%%%%% 
\begin{figure}
\begin{center}
\includegraphics[width=0.8\textwidth]{images/rosa-gsi-table-crop.pdf}
\end{center}
\caption[Assignment table by RoSA genotype]{\footnotesize Assignment table
like that in Fig.~\ref{fig:gsi}b in the paper, but with numbers according to genotypes
at the RoSA}
\label{fig:rosa-gsi}
\end{figure}
%%%%%%%%%%%%%%%%%%%%
