%!TEX root = main.tex


Genetic methods have become an essential component of ecological investigation
and conservation planning for fish and wildlife. Among these methods is the use
of genetic marker data to identify individuals to
populations, or stocks, of origin. More recently, methods that involve genetic
pedigree reconstruction to identify relationships between individuals within populations
have become common. We present, here, a novel set of
multi-allelic microhaplotype genetic markers for Chinook salmon which provide unprecedented
resolution for population discrimination and relationship identification for a rapidly and
economically assayed panel of markers. We show how this set of microhaplotypes
provides definitive power to identify all known lineages of Chinook salmon in
California. The inclusion of genetic loci that have known associations with phenotype
and that were identified as outliers in examination of whole genome sequence data,
allows resolution of stocks that are not highly genetically differentiated but
are phenotypically distinct and managed as such. 
This same set of multiallelic genetic markers have ample variation to
accurately identify parent-offspring and full-sibling
pairs in all California populations, including the genetically depauperate winter-run
lineage.
Validation of this marker panel in coastal salmon populations not previously studied with
modern genetic methods, also reveals novel biological insights, including
the presence of a single copy of a haplotype for a phenotype that has not
been documented in that part of the species range, and a clear signal of mixed
ancestry for a salmon population that is on the geographic margins of the 
primary evolutionary lineages present in California.
