%!TEX root = supplement.tex

\section{Winter-run-associated polymorphism (WRAP) methods and results\label{sec:wrap-methods}}

We used the whole genome sequencing data from \citet{thompson2020complex} to seek variants with
large allele frequency differences between winter-run Chinook salmon and all the other
Chinook salmon ecotypes in the Central Valley of California (CCV).  Because winter-run Chinook are
already highly differentiated from all others, we did not pursue an association study as used
for identifying the late-fall-run associated variants.  Rather, we first identified regions with a high
density of variants with large allele frequency differences between winter-run and non-winter-run
fish in the CCV\@. Subsequently we identified SNPs within those regions with particularly
large allele frequency differences.  This approach was taken to avoid targeting single SNPs
with large allele frequency differences resulting from sampling variation, which was a
concern because we had whole genome sequencing data from only 16 winter run fish.

More specifically, we calculated allele frequencies for 16 winter-run fish and 84 non-winter-run
fish from the  \citet{thompson2020complex} variant data VCF files using ANGSD version~0.921.
Sites were retained if at least 62.5\% of samples in each group had read data (10 of 16 for the
winter run and 50 of 80 for the non-winter run), resulting in 7,295,001 SNPs for downstream
analysis.  We first investigated the distribution throughout the genome of $|d|$, the absolute difference
of the alternate allele frequency between the two groups  (Fig.~\ref{fig:wrap-absdiff}). This revealed
that many loci had large values of $|d|$, but there were several
regions in the genome, in particular, with prominent peaks in allele frequency difference and one
or more SNPs apparently fixed for alternate alleles between the two sample groups.  

To leverage information from multiple SNPs to identify regions in the genome with large
allele frequency differences, we calculated the fraction of SNPs with $|d| > 0.5$ that also
had $|d| > 0.9$ within non-overlapping 100 kb sliding windows throughout the genome. This metric indicated
several prominent peaks.  We focused on all of those
100 kb windows in which more than 12.4\% of sites with $|d|>0.5$ also had $|d|>0.9$ which were
also adjacent to at least one window in which $>10\%$ of sites with $|d|>0.5$ also had $|d|>0.9$ 
(Fig.~\ref{fig:wrap-slide-window}).  Within the windows found on four different chromosomes using
the above criteria, we then attempted to design amplicons to type the SNPs at a subset of the sites
within each window.  We chose all sites with $|d|>0.975$, as well as the 8 SNPs on each chromosome
with the highest values of $|d|$.  

\comment{ECA still has to add the WRAP marker-exploration steps here}

